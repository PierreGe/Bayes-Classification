\documentclass[a4paper,10pt]{article}
\usepackage[french]{babel}
\usepackage[utf8]{inputenc}
\usepackage[left=2.5cm,top=2cm,right=2.5cm,nohead,nofoot]{geometry}
\usepackage{url}
\usepackage{graphicx}
\usepackage{float}
\usepackage[colorinlistoftodos]{todonotes}
\usepackage{hyperref}
\usepackage{amssymb}

\linespread{1.1}



\begin{document}

\begin{titlepage}
\begin{center}
\textbf{\textsc{UNIVERSIT\'E DE MONTR\'EAL}}\\
%\textbf{\textsc{Faculté des Sciences}}\\
%\textbf{\textsc{Département d'Informatique}}
\vfill{}\vfill{}
\begin{center}{\Huge Rapport : Devoir1 }\end{center}{\Huge \par}
\begin{center}{\large Pierre Gérard}\end{center}{\Huge \par}
\vfill{}\vfill{} \vfill{}
\begin{center}{\large \textbf{IFT3395-6390 Fondements de l'apprentissage machine}}\hfill{\\Pascal Vincent, Alexandre de Brébisson et César Laurent}\end{center}{\large\par}
\vfill{}\vfill{}\enlargethispage{3cm}
\textbf{Année académique 2015~-~2016}
\end{center}
\end{titlepage}

%\begin{abstract}
%Ce rapport présente ...
%\end{abstract}


\tableofcontents

\pagebreak

\section{Partie Théorique : Estimation de densité}

\subsection{Densité avec des fenêtres de Parzen a noyau Gaussien isotropique}

a) $\mathcal{N}_{x^{(i)}, \sigma^{2}} = \frac{1}{(2 \pi)^{d/2} \sigma^{d}} e^{ \frac{-1}{2} \frac{d(x^{i},x)^{2}}{\sigma^{2}} } $

\todo{Paramètre h}

b) Si on choisit de prendre tous les points du voisinages
$ p(x) =  \frac{1}{n} \sum_{i=1}^{n} \frac{1}{smoothing^{d}} \mathcal{N}_{x^{(i)}, \sigma^{2}} $

c) 

Hyper-paramètre: $\sigma$

Paramètre : On mémorise l'ensemble des données d'entrainement.
On considère cette méthode comme non-paramétrique car le nombre de paramètre varie avec la taille de l'ensemble de données.

\todo{Taille des paramètres}

d)

\subsection{Densité paramétrique avec Gaussienne isotropique}

a) $p(x) = \frac{1}{(2 \pi)^{d/2} \sqrt{\mid{\Sigma} \mid}} e^{\frac{-1}{2} (x- \mu)^{T} \Sigma^{-1} (x- \mu) }$


$\mu$ est le est le vecteur des moyennes, de dimension d.
$\Sigma$ est la matrice de covariance, de dimension 1.



b) sigma devient un paramètre ?

c)

Paramètres :
\begin{itemize}
	\item $\mu$ est le est le vecteur des moyennes, de dimension d.
	\item $\Sigma$ est la matrice de covariance, de dimension 1.
\end{itemize}

Pas d'hyper-paramètre.


\subsection{Densité paramétrique avec Gaussienne diagonale}

a) $p(x) = \frac{1}{(2 \pi)^{d/2} \sqrt{\mid{\Sigma} \mid}} e^{\frac{-1}{2} (x- \mu)^{T} \Sigma^{-1} (x- \mu) }$
\begin{itemize}
	\item $\mu$ est le est le vecteur des moyennes, de dimension d.
	\item $\Sigma$ est la matrice de covariance, de dimension d.
\end{itemize}

b) \todo{video larochelle covariance zero}

c)

d)


\section{Partie Théorique : Classifieurs de Bayes}

\subsection{Classifieur de Bayes obtenu avec des densités paramétriques Gaussiennes diagonales}

a)

b)

c)

\subsection{Classifieur de Bayes obtenu avec des fenêtres de Parzen à noyau Gaussien isotropique}

a)

b)


\subsection{Classifieur de Bayes obtenu avec des fenêtres de Parzen à noyau Gaussien isotropique}

a)

b)



\end{document}
